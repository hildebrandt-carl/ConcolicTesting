\section{conclusion}
In this paper, we first presented and implemented a framework for doing concolic testing on LLVM bitcode. This involves two LLVM passes and a python program. The first LLVM pass
generates symbolic constraints for a program statically. The second LLVM pass inserts prints statements that generates program traces of the constraints evaluated on that
execution. These constraints are then passed to the python program which uses an SMT solver to find new inputs that satisfies new constraints of a program.

When compared against the constraints generated by the static pass, the dynamic pass along with constraint solver generate fewer inputs that still satisfy all paths. This
is due to the fact that the dynamic pass is flow sensitive unlike the static pass. If we were to generate input based solely on the static constraints, we would have to generate an
input for every satisfiable conjunction of the constraints to ensure we hit all paths. 