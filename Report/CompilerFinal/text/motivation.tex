\section{Motivation}
As software systems get larger, symbolic execution is less likely to be able to perform an efficient, correct analysis to generate inputs that tests all paths of execution.
At the same time, concrete execution using a fuzzing tool that generates random input will not be able to exhaustively test these systems. There is no guarantee that 
a fuzzer can exercise all paths in a program without generating all possible input. 

If a fuzzer were to analyze Code Snippet 1, it could generate input that could easily exercise the path $(input>5)\land (input> 6)$ with any input greater than 6. To execute the
path $(input>5) \land !(input>6)$, there is only one input, namely 6, that would satisfy this constraint. The randomness of the fuzzer doesn't guarantee this path will be found until
all inputs have been tested. If this input is an int and this were a C program, this in the worst case situation could take $2^{32}$ executions. However, a symbolic execution tool could solve the constraints of the path,
and simply return three tests that would satisfy all paths of the code snippet, for example: model($(input>5)\land (input> 6)$) = 7, model($(input>5)\land !(input> 6)$) = 6, and model($!(input>5)$) = 0.
\begin{lstlisting}[caption={Possible Error Code}, language=C, label=snip:motivation]
...
if(input>5){
    ...
    if(input>6){
        ...
    }
    else{
        error();
    }
}
...
\end{lstlisting}

A problem arises with scalability when using symbolic execution tools. If a symbolic execution tool were to analyze a large codebase, the constraints to execute path with compound and may reach a point where solving the constraints may be
more computationally expensive than fuzzing the program. If the paths that are computationally complex to solve are executed with a reasonable high probability,
then a fuzzer can be used to generate random input that will hit this path instead.

Using these two types of tools together can lead to avoiding both situations described above. In this paper we design a workflow that would allow for concolic testing of LLVM bitcode. Concolic testing uses concrete execution combined with symbolic constraints to generate new inputs which explore a program.