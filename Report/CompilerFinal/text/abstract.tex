\begin{abstract}
  Concolic Execution is a testing strategy that utilizes both concrete
  testing and symbolic execution to make up for the shortcomings of each
  technique. Concrete execution is as fast as the program in question. 
  However, if inputs are being generated at random, it may take far
  too many tests to satisfy all paths of execution. Symbolic execution 
  can find all paths of execution. However, this can be computationally 
  expensive. Concolic execution leverages one against the other. Symbolic 
  execution can find the paths in which concrete testing has difficulty
  reaching. Concrete execution can execute the hard to solve paths. Many
  concolic execution tools target a specific language. This can lead to
  time spent re-targeting tools to other languages. LLVM is a compiler
  infrastructure that allows for creation of compilers that produce 
  the same intermediate representation, LLVM bitcode. In this paper, we 
  design a concolic execution strategy that can be used on LLVM bitcode.
  This will allow for a concolic execution tool that can be used on
  languages that can be compiled into LLVM bitcode.
\end{abstract}