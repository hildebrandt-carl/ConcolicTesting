\section{Introduction}

Symbolic execution and concrete execution are two testing techniques
that can be used to find and test execution paths through a program.
Symbolic execution finds the constraints of a program to discover paths
of through the program. In order to test these paths, it generates the constraints
that need to be satisfied to execute it. It then passes these constraints
to a tool, like the SMT solver Z3 \cite{10.1007/978-3-540-78800-3_24}, which
decides if these constraints are satisfiable. If it is satisfiable, it can model
this solution with a input that with execute the path in question. Solving
these constraints can quickly become difficult as the constraints become
more and more complex. As a result, using symbolic execution tools can become
infeasible with complex programs. 

On the other end of the spectrum, there is concrete execution. Unlike symbolic
execution, concrete execution is relatively inexpensive. The test is as 
expensive as the program in question. This allows
for running many tests to exercise as many paths as possible. Often the input
for automated concrete execution is randomly generated. If the probability of
going down any path is fairly uniform, testing all paths is fairly simple. 
However, if the probability of hitting certain paths is low, it can be 
difficult to exercise all paths. However if combined correctly, these tools
can compliment each other resulting in a more accurate yet cheap automated test tool. 

Another issue common in many software testing tools is that the tool can only 
target a specific language. This can lead to several tools that may 
accomplish the same task, only for different targets. The scope of the 
tool can be expanded if it targets a more general language. LLVM is
a compiler infrastructure that provides the tools for developers to
make compilers for their own languages or existing languages. LLVM
compilers produce an intermediate representation known as LLVM bitcode.
Tools that target LLVM bitcode should then be able to operate on the
various languages that have a compiler that produces LLVM bitcode.

The LLVM infrastructure allows for various "passes" to be run on 
LLVM bitcode. These passes can perform static analysis, such as 
dependency analysis, which finds dependencies in memory usage. They
can also transform code by adding instructions or optimizing it with
passes like constant propagation \cite{llvm}. We create two passes, one that statically
collects information about the program at compile time, and one that 
instruments code to output information dynamically at execution time.
Using the information generated by these passes, we can generate input that
will explore the paths of a program more efficiently than symbolic execution
and more exhaustively than concrete execution.

We present a framework to do concolic testing on LLVM bitcode and show
the workflow of using this framework to exercise the paths of execution
of a program. This framework uses the two LLVM passes written by the authors.
The first pass is a static pass run at to generate the constraints of a program.
The constraints are used to ensure we exercise all paths. The second pass is a dynamic pass
that instruments the code that instruments the bitcode to report the trace of the program
with respect to the constraints it reaches and whether or not the input satisfies them.
Using these traces, we can generate new input that follows paths not executed
by past input. More specifically this paper has the following contributions:

\begin{enumerate}
    \item We design a concolic test engine targeted at LLVM bitcode.
    \item We design and implement a dynamic LLVM pass that instruments a program to report a trace of that programs execution.
    \item We design and implement a static LLVM pass that analyzes and displays all constraints in a program.
    \item We design and implement a python program that uses an SMT solver to solve a given set of constraints.
\end{enumerate}